\section{Bilan \& perspectives}
Ce projet a finalement abouti à l'élaboration de deux services de reconnaissance d'entités nommées~: un premier prototype fonctionnel et un second service, plus industriel. Ils montrent tous les deux des résultats encourageants, et qui serviront tant aux rédacteurs des publications qu'à l'équipe en charge du référentiel des métadonnés statistiques. Cela m'a de plus permis de préciser mon projet professionnel, en explorant à la fois les thématiques de traitements de langage naturel, de développement web ainsi que la thématique DevOps.

\subsection{Le livrables et les résultats actuels}
Les livrables du projet, disponibles sur \href{https://github.com/Mardelor?tab=repositories}{Github}, sont au nombre de trois~: les deux services de REN et ce présent rapport.
\newline

Le premier service permet à ce jour de repérer les concepts de la base RMéS, et de générer un rendu XML et un rendu HTML d'une même publication. Un pipeline GitLab permet déjà de tester automatiquement le programme sur n'importe quelle publication figurant dans la base de données au format XML. Il donne une bonne vision de ce que l'on peut faire en TLN sur les publications de l'Insee, même s'il a quelques lacunes.
\newline

Une analyse du programme sur environ 650 publications, dont chacune fait trois à quatre pages, a été menée. Il s'agit de rechercher quels concepts sont repérés dans le corpus, sous quelles formes et à quelle fréquence. On constate que la moitié des concepts est repérée sur les 1162 qui sont présents dans la base RMéS. Pour une majorité d'entre eux, ils ne sont trouvés moins de dix fois dans le corpus. Pour les concepts généraux en revanche, comme «~salaire~», «~commerce~» ou «~productivité~», on retrouve plusieurs formes et avec un nombre d'occurrences allant de quelques dizaines à quelques centaines.

Il est difficile de déterminer si les concepts qui n'apparaissent pas ne sont tout simplement pas mentionnés ou si cela est dû aux lacunes du pipeline. Ce que l'on peut en conclure en revanche, c'est que les concepts généraux sont cités avec des formes diverses. Cela pointe le fait que certains concepts mériteraient peut-être d'avoir des sous-concepts, plus spécifiques.
\newline

Le second service est plus abouti~: construit pour pouvoir être utilisé sur le Web, il ne permet cependant pas, à l'heure actuelle, d'effectuer les tests sur les publications. Il dispose en revanche d'une interface pour soumettre du texte au pipeline, et ainsi rapidement vérifier si un concept est reconnu dans un certain contexte. Cette interface inclut également l'affichage des dépendances, permettant de résoudre les dysfonctionnements plus rapidement. Il est également adressable en HTTP et peut fournir le résultat du pipeline en JSON. Je n'ai malheureusement pas eu le temps de mettre en place un pipeline GitLab. J'ai cependant créé une image Docker sur laquelle l'application peut s'exécuter.
\newline

Bien que je n'aie pas pu finaliser la seconde application, j'ai pris soin de laisser une documentation, afin que d'autres puissent faire aboutir le second service. 

\subsection{Perspectives d'améliorations}

Plusieurs améliorations peuvent être apportées au projet. Il manque en effet un élément de comparaison des deux bibliothèques SpaCy et Stanford Core NLP sur les données de l'Insee. On pourra pour cela développer un simple client du service développé avec SpaCy et qui effectue les mêmes statistiques. Il suffit pour cela de reprendre le module de test du projet \textit{Concept-tagger} et d'y ajouter un client HTTP.
\newline

Une autre amélioration est, bien entendu, l'implémentation du service utilisateur. Cela donnerait un outil pour les rédacteurs, ainsi qu'un moyen d'enrichir la base RMéS. Dans une optique plus large, il apporterait je pense un vrai plus aux agents de l'Insee qui souhaitent en savoir davantage sur les applications possibles des Data science. Il servirait également à l'Insee pour créer une première base d'apprentissage et donc à valoriser ses données.
\newline

Il manque également au projet que j'ai baptisé \textbf{InspaCy} son intégration sur la plateforme Innovation. Le travail qu'il reste à accomplir est minime et j'espère avoir l'occasion de le finir avant la fin du stage. C'est un travail important puisque cela donnerait une visibilité interne au projet. L'interface développée donne un bel aperçu du potentiel du traitement du langage naturel (voir section \ref{section 2.1.4}).
\newline

Il y a enfin deux aspects que j'aimerai améliorer dans ce projet. Le premier est de combler les lacunes du premier pipeline, en générant le dictionnaire de lemmes évoqué à la partie \ref{section 2.2.5}. Le second est de tester une autre méthode pour repérer les entités nommées. Cette méthode se base sur les résultats du traitement \textit{Dependencies}~: il s'agit, à partir de l'arbre syntaxique des phrases, de repérer les groupes nominaux, et de les comparer aux libellés d'entités nommées. Si le lemme du nom principal ainsi que les lemmes d'un certain nombre de qualificatifs, disons au moins 50 \% correspondent, alors le groupe nominal est annoté. La méthode repère de façon plus large les entités que le pattern décrit \autoref{fig:schema-tokensregex}. Cela permettrait de ne manquer aucune mention de concept, et pourrait faciliter encore la génération de données d'entraînement.
\newline

Beaucoup d'améliorations peuvent donc encore être apportées au projet. Bien que je n'aie pas pu aboutir à une comparaison des deux services TLN, ce projet m'a tout de même beaucoup apporté.

\subsection{Apport dans mon projet professionnel}
\subsubsection{La vision apportée vis-à-vis de mon projet professionnel}

Travailler avec la DIIT et Franck Cotton pendant ces six mois a été très enrichissant pour moi. Cela m'a fait réfléchir sur le contexte dans lequel je souhaite travailler, ma relation avec les autres, et les sujets sur lesquels j'aimerais travailler.
\newline

Le stage a tout d'abord renforcé ma confiance en moi. Être accueilli dans une structure de plus de 6000 agents a d'abord été intimidant. J'ai en effet eu en début de stage des difficultés à me placer par rapport à mon équipe et ma mission, car ma mission n'est pas celle de l'équipe. Aller voir les bonnes personnes susceptibles de m'aider dans ma tâche, oser relancer les demandes restées sans réponse, tout cela a d'abord été déstabilisant. C'est une difficulté que j'ai surmontée grâce à la DIIT qui m'a bien aidée dans ma démarche. J'ai en effet été inclus dans les sujets traités par l'équipe comme la gestion de la plateforme Innovation. Cela a été très stimulant, d'autant plus que certains sujets avaient été abordés dans mes cours. Les incubations (voir section \ref{section 2.2.2}) en particulier étaient très intéressantes. Découvrir des projets en développement et réfléchir en groupe sur leur intégration dans la plateforme m'a beaucoup apporté. Cela confirme mon envie de travailler en équipe projet, comme je l'ai fait auparavant dans les associations de l'école.
\newline

Ce stage m'a aussi apporté une vision métier~: l'administration d'une plateforme \textit{CaaS} (voir partie \ref{section 2.2.1}). Ajouter de nouveaux services, remplacer ceux qui ne sont plus maintenus ou encore expérimenter les nouveaux et adapter l'architecture d'une telle plateforme, tous ces sujets m'ont beaucoup intéressé; d'une part car cela permet d'explorer tout un tas d'applications~: services d'authentification centralisés, environnements de développement, services de stockage réparti, outils pour du machine learning, etc. D'autre part cela permet des interactions avec de nombreuses personnes ayant des profils variés. Ces échanges ont été enrichissants dans ma recherche des domaines dans lesquels je souhaite travailler.  
\newline

J'ai désormais acquis la certitude de vouloir travailler dans le domaine Recherche et Développement. L'étude d'articles de recherche et leur mise en pratique dans des problèmes concrets est un aspect que j'ai beaucoup apprécié dans le projet. On a à la fois une vision sur les nouveautés technologiques et sur les besoins pratiques des entreprises. C'est pourquoi je m'oriente aujourd'hui vers la recherche en entreprise et leurs services de R\&D.
\newline

\subsubsection{Les compétences apportées}

Ces six mois à la DIIT m'ont enfin apporté davantage de compétences techniques et transversales. Je suis aujourd'hui en mesure de développer un projet orienté DevOps~: la maîtrise des paradigmes implémentés dans \textbf{Docker} et \textbf{GitLabl-CI/CD} me donne toutes les connaissances pour expérimenter rapidement sur des nouveaux projets. J'ai également pu en apprendre davantage sur des bibliothèques comme \textbf{Flask} ou \textbf{React}, standards récents pour la mise en place d'applications webs. Ce projet m'a bien entendu donné une première expérience dans le domaine du Traitement du langage naturel, et plus largement dans l'utilisation des Data sciences dans des contextes très spécifiques comme celui de l'Insee.
\newline

J'ai également pu améliorer la qualité de mes présentations orales. Le sujet est en effet assez vaste et utilise du vocabulaire spécialisé. C'est pourquoi j'ai dû travailler la présentation de mes résultats plusieurs fois afin de les rendre accessibles. J'ai d'ailleurs été aidé par les fonctionnalités offertes par \href{https://revealjs.com}{Reveal.js} \cite{}, une bibliothèque Javascript permettant de faire des présentations interactives. Structurer mon discours, ajuster ma gestuelle et le volume de ma voix~: les quatre présentations effectuées à l'Insee m'ont beaucoup aidé.
\newline

\subsubsection*{Conclusion}

J'aborde aujourd'hui l'entrée dans le monde du travail avec confiance et sérénité, et ce en grande partie grâce à ce stage. Je pense avoir toute les capacités pour trouver ce qui correspond dans le domaine de la Recherche et Développement.
